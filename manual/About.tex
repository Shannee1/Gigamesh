\chapter*{Preface}
\addcontentsline{toc}{chapter}{Preface}

\GigaMesh is a software framework for processing large meshes. 
It is mainly running on linux operating systems, and a windows version is preparation. \GigaMesh is able to read established mesh formats like {\tt Stanford PLY} or {\tt Wavefront OBJ}. It has a graphical user interface and features the possibility of a fast and easy display of large 3D models. The mesh can be polished  to result in a perfect 2D manifold,  true to scale screenshots and  image sequences for video output can be achieved easily, and profile lines can be exported as vector graphics.
Command line tools are provided for intense computing tasks like the application of 3D filters for pattern recognition. 

\GigaMesh is developed by Hubert Mara during his Ph.D. thesis in collaboration with the {\em Heidelberg Academy of  Sciences and Humanities}. Its first intension was the visualization of cuneiform script but soon developed into a tool for a wide range of objects in Cultural Heritage. Contributions concerning the fill holes algorithm (using the library {\tt libpsalm})
and the unwrapping algorithm are made by Bastian Rieck. Other contributions are due to Andreas Beyer, Robert Kreuzer, and Philipp Stekl.
%A precise algorithm for surface integral invariants was added by Philipp Stekl. 
The first version of this manual was written by Anja Sch\"afer. Everyone who is working with the program and its manual is welcome to make suggestions for  improvements.

The software name \GigaMesh derives from the Epic of Gilgamesh (see the recent translation \cite{maul05a}), which is among the earliest known works of the literature, written in cuneiform script. Shorten the first syllable in \GigaMesh by the letter "l" associates the SI prefix G = giga = $10^9$ (a billion), and thereby describes the capability of dealing with large meshes. Intentionally it is developed to improve the reading of cuneiform tablets, so  the name is already program. The \GigaMesh logo is derived from a cuneiform known as {\em kaskal = path {\em or} way}, combined with a red highlighted ball in the style of the golden ball of our group logo.

We thank Willi J{\"a}ger, Stefan M. Maul and Stefan Jakob  for their support and granting access to the collections of the \emph{Assyriologie, Seminar f\"{u}r Sprachen und Kulturen des Vorderen Orients} of the Heidelberg University.
The 3D scanners used for the acquisition of the presented data were provided by the \emph{\href{http://www.haw.uni-heidelberg.de/}{Heidelberg Academy of  Sciences and Humanities}} (HAW) and
the \emph{\href{http://www.mathcomp.uni-heidelberg.de/}{Heidelberg Graduate School of Mathematical and Computational Methods for the Sciences}} (HGS MathComp).
This work is part of the \emph{\href{http://ipp.uni-hd.de/}{IWR Pioneering Projects}} (IPP) and partially funded by the HGS MathComp -- DFG Graduate School 220. 

\vspace{0.5 cm}

Susanne Kr\"omker


Heidelberg, August 2011
\newpage
\thispagestyle{empty}