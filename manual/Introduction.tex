%==============================================================================
%==============================================================================
\chapter{Introduction}
\label{chap:intro}

\GigaMesh is a software framework for displaying, processing and visualizing large meshes representing 2D surfaces typically acquired with 3D scanning techniques.  

Its core piece mainly base on the {\em Multi Scale Integral Invariant} (MSII) filtering technique~\cite{mara12b}.
Integral invariants~\cite{pottmann09a,huang06a} using volume and surface descriptors can be computed for feature detection in large triangular meshes.
Results of the computations are stored in the so-called feature vectors. 
Visualizations of these feature vectors use standard metrics known from image processing like auto-correlation or distance measures.
These metrics can be visualized by false-coloring the 3D models using various colorramps.

\GigaMesh can be operated with or without a graphical user interface (GUI). 
It is based on OpenGL and Trolltech Qt, and features the possibility of a fast and easy display of large 3D models. 
The mesh can be polished  removing measurement errors like solo vertices, singularities and holes, resulting in an oriented filled and perfect mesh of the quality of a 2D manifold. 
The resulting mesh can be saved and afterwards imported by other software tools. Identified features can be extracted as points or polygonal lines.
\GigaMesh offers the possibilities to create orthographic or perspective normalized viewings, 
true to scale screenshots, to prepare image sequences for videos and interactive Adobe Flash and Apple QuickTime output, 
and to export polygonal lines as vector graphics in {\tt *.svg}~\cite{eisenberg02a} format.
There exist several editors for this vector format such as the \emph{Inkscape}\footnote{\url{http://inkscape.org} -- last visited on 7.August 2013.}, 
which is highly recommended as it supports {\tt *.svg} natively and it is freely available for all major operating systems.

\newpage
%==============================================================================
\section{Operating system}
\GigaMesh is currently running mainly on Linux. 
The version for windows is expected to be released, when Qt 5.1 becomes available. 
An unstable Macintosh binary exists missing some of the functions.
%If you work on windows systems ask for a boot CD with a  standalone executable.


%==============================================================================
\section{Mesh formats}
\GigaMesh supports the by now most established 3D formats for mesh data used in metrology. 
They can be opened within the graphical user interface. 
A file browser and appropriate filter masks then show the following data due to their filename extensions:
\begin{itemize}
	\item Stanford PLY -- ASCII and binary with big and little endianness\footnote{Byte order.}
	\item Wavefront OBJ -- ASCII 
	\item Pointclouds written in ASCII having file extension {\tt *.txt} or {\tt *.xyz}
\end{itemize}

Reading and writing of these formats is due to filename extension. 
Stanford PLY file format is highly recommended. 
Meta information stored when saving PLY-files with \GigaMesh are displayed when using these files later on with \GigaMesh. 
Other programs skip these headers.

There are also two types of proprietary formats (ASCII): 
It is possible to (re-)import texture maps as {\tt *.tex} and feature vectors as {\tt *.mat} files which have been generated in previous runs of the program.

\paragraph{Pitfall:} due to new Operating system libraries reading and writing {\bf ASCII} files is prone to errors
due to different characters for floating point values.
Many regional settings use a dot (''{\tt .}'') for the {\bf decimal point}, 
while Operating system having e.g.~a German localization use a comma (''{\tt ,}'').
This affects also import/export from/to software packages\!
