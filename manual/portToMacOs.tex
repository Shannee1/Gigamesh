\documentclass[%
a4paper,						% alle weiteren Papierformat einstellbar
%landscape,						% Querformat
12pt,							% Schriftgre (12pt, 11pt (Standard))
%BCOR1cm,						% Bindekorrektur, bspw. 1 cm
%DIVcalc,						% fhrt die Satzspiegelberechnung neu aus, s. scrguide 2.4									
%twoside,						% Doppelseiten
%twocolumn,						% zweispaltiger Satz
parskip=half*, 					% Absatzformatierung s. scrguide 3.1
headsepline,					% Trennline zum Seitenkopf	
%footsepline,					% Trennline zum Seitenfu
%titlepage,						% Titelei auf eigener Seite
headings=small,					% berschriften etwas kleiner (smallheadings)
%index = totoc,					% Index im Inhaltsverzeichnis
%listof = totoc,				% Abb.- und Tab.verzeichnis im Inhalt
%bibliography= totoc			% Literaturverzeichnis im Inhalt
%bibtotoc = true,
%bibtotocnumbered = true,
%bibliography=totocnumbered,
%draft								% berlangen Zeilen in Ausgabe gekennzeichnet
]
{scrartcl}

\usepackage[T1]{fontenc}
\usepackage[ngerman]{babel}

%\usepackage{ucs}
\usepackage[utf8]{inputenc}	
\usepackage[babel]{csquotes}
\usepackage{url}
\usepackage{color}
\usepackage{listings}
\usepackage{footnote}
\usepackage[onehalfspacing]{setspace}
\usepackage[bottom,hang]{footmisc}
\setlength{\footnotemargin}{1pt}
\deffootnote%
{1em}% Einzug des Funotentextes; bei dreistelligen Funoten evtl.vergrern
{1em}% zustzlicher Absatzeinzug in der Funote
{%
\makebox[1.5em]% Raum fr Funotenzeichen: ebenso gro wie Einzug des FN-Textes
[c]% Ausrichtung des Funotenzeichens: [r]echts, [l]inks
{\textsuperscript{\thefootnotemark}}% Funotenzeichen
}


% Title Page
\title{HowTo:\\ Port GigaMesh\\ to MacOs Lion}
\author{Christian Seitz}
\definecolor{lightgrey}{rgb}{0.95,0.95,0.95}

\lstset{ %
  backgroundcolor=\color{lightgrey},  % choose the background color; you must add \usepackage{color} or \usepackage{xcolor}
  basicstyle=\footnotesize,       % the size of the fonts that are used for the code
  frame=single,
  literate={~} {$\sim$}{1}
  }

\begin{document}
\maketitle
\vspace{5cm}
\tableofcontents
\newpage

\section{Vorbereitung}
\subsection{XQuartz}
Da leider der native X-Server unter MacOs einige Funktionen nicht bietet, muss der alternative X-Server \glqq XQuartz\grqq ~installiert werden (download hier: \url{http://xquartz.macosforge.org})

Nach der Installation muss man sich ab- und wieder anmelden, um den neuen X-Server in Betrieb zu nehmen.

\subsection{Xcode}
Für die Benutzung der Apple-Programmierumgebung auf die Homebrew aufbaut, muss Xcode installiert sein. Über den Appstore kann es kostenlos heruntergeladen werden, man benötigt allerdings eine Apple-ID.

Zusätzlich muss man, sobald Xcode installiert ist, es starten und über die Einstellungen (im Menü oben auf Xcode $\rightarrow$ Preferences) im Reiter \verb Downloads ~die \glqq Command Line Tools\grqq ~installieren.

\subsection{Homebrew}
Um einige der Linuxlibraries benutzen zu können, braucht man \glqq Homebrew\grqq, das dies möglich macht:

Zuerst ein Terminal öffnen (\texttt{Programme $\rightarrow$ Dienstprogramme $\rightarrow$ Terminal} -- alternativ in Spotlight (die Lupe oben rechts in der Ecke des Monitors) \glqq \verb terminal \grqq ~eintippen)

Hier folgenden Code einfügen:
\begin{lstlisting}
ruby -e "$(curl -fsSkL raw.github.com/mxcl/homebrew/go)"
\end{lstlisting} 
Es wird im Verlauf der Installation nach einer Bestätigung gefragt, dies bitte mit \texttt{Enter} durchführen.

Nun folgen einige Modifikationen an den Umgebungsdateien, damit beim Erstellen (kompilieren) des Programms zuerst die mit Homebrew installierten Libraries verwendet werden.

Dazu muss jetzt die Datei \verb /private/etc/paths ~verändert werden\footnote{Quelle: \url{https://gist.github.com/1669348}}. Im Terminal gibt man dazu ein:
\begin{lstlisting}
nano /private/etc/paths
\end{lstlisting}
und bestätigt mit Enter.

Der Inhalt der Datei sollte folgendermaßen aussehen:
\begin{lstlisting}
/usr/bin
/bin
/usr/sbin
/sbin
/usr/local/bin
\end{lstlisting}

sie muss korrigiert werden, so dass die Zeile \verb /usr/local/bin ~vor \verb /usr/bin ~liegt, also folgendermaßen:

\begin{lstlisting}
/usr/local/bin
/usr/bin
/bin
/usr/sbin
/sbin
\end{lstlisting}
Mit \verb STRG+X ~wird die Datei geschlossen und die Frage, ob gespeichert werden soll, mit \verb y ~beantwortet. 

\section{Die Installation}
Nun können die Pakete mit Homebrew installiert werden. Im Terminal gibt man dazu folgendes ein:
\begin{lstlisting}
brew install git glew libtiff qt cairo
\end{lstlisting}
Nun werden die Libraries heruntergeladen und kompiliert, das kann eine ganze Weile dauern.

Ist der Vorgang abgeschlossen, kann man sich GigaMesh klonen\footnote{Git wurde über Homebrew mit installiert. Hat gitte Ihren Public-Key für SSH installiert, so müssen Sie noch die Datei \tt{id\_rsa} in den Ordner \tt{$\sim$/.ssh/} kopieren, damit git richtig funktioniert. Für Fragen zu git wenden Sie sich an ?}.
Dazu wechselt man zunächst im Terminal in den Ordner in den man GigaMesh klonen möchte, hier beispielsweise den Desktop:
\begin{lstlisting}
cd ~/Desktop/
\end{lstlisting}

Mit folgendem Befehl wird ein Ordner GigMesh auf dem Desktop angelegt und die Daten vom git-Server geholt:
\begin{lstlisting}
git clone gitosis@gitte.iwr.uni-heidelberg.de:GigaMesh
\end{lstlisting}

Nach Abschluss wechselt man in das GigaMesh-Verzeichnis (\texttt{cd GigaMesh}) und führt dort folgende Befehle aus:
\begin{lstlisting}
qmake && make
\end{lstlisting}
 
Nun wird Gigamesh kompiliert. Nach dem kompilieren kann gigamesh mit \verb ./gigamesh ~gestartet werden.
 
\section{Bemerkungen}
Nachdem ein \verb .ply ~geladen wurde, kann es vorkommen, dass es nicht gleich korrekt angezeigt wird. Dann nur das Mausrad drehen oder mit der linken Maustaste das Objekt drehen.

VORABVERSION -- ZUM TESTEN!
\end{document}